% Documentation for tickets system

\NeedsTeXFormat{LaTeX2e}[1995/12/01]
\documentclass[a4paper,12pt]{article}

\title{MtGox API - Tickets system}

\author{\copyright~Copyright 2012, MtGox. All rights reserved.}

\date{3 August 2012 (WIP)}

\begin{document}

\maketitle

\tableofcontents

\section{Introduction}

As MtGox is currently phasing out the \emph{redeemable codes}, a new method
for partners to process funds from customers is required.

MtGox has been working with its partners on a new method that would prevent
user errors, make the whole process of withdrawing funds easier and offer more
security against money laundering and other abuses.

Please note that the features described in this document are \emph{only}
available to partners in contract with MtGox.

\textbf{Please note that this documentation and the described features are
currently \emph{work in progress}, and as such may change in ways not
described in this document at anytime. Please contact your KAM at MtGox to
obtain the latest version of this document}

\subsection{Requirements}

As various customer details are made available to partners within the scope
of this feature, each partner is required to ensure that conditions applying
to the transmitted data are respected.

Especially, MtGox's Privacy Policy\footnote{Available at
https://mtgox.com/privacy\_policy} specifies the scope under which the shared
details can be used.

\subsection{Process}

When a MtGox customer wishes to withdraw funds from MtGox, he will be given
different choices. Should the customer choose a partner using this withdraw
method, the customer will be redirected to the partner after confirming his
withdraw request and accepting the terms explaining that his account's details
will be made available to the partner. Note that it is possible for partners
to require that only verified users can use this system.

When the customer confirms his withdraw, a ticket is created holding the
withdrew amount, and that ticket is forwarded to the partner through the user
himself. Two key elements are provided: the ticket id (passed via GET to the
URL provided by the partner) and the ticket key (passed via POST). The partner
must ensure that the ticket key is not to be found in the URL in order to
ensure access to the browser's history will not allow access to this
information.

Once on the partner's website, the customer has the ability to choose the
destination of the withdrew funds. From the ticket id and key, the partner
can access various details through the API:

\begin{itemize}
\item Basic account details, including the user name, email and account ID
\item Withdraw amount and currency
\item Date of request, and expiration date
\item Advanced account details, specifying if the user is verified and how
\end{itemize}

Once the partner has confirmed the transfer with the user and verified that
everything is in order, the partner should confirm the transaction to the
user, and must provide a link back to MtGox.

Before starting to process the transaction but after verifying that all the
required details are provided, the partner must settle the ticket, thru
crediting his own MtGox account and ensuring that the ticket will not expire.

At this step, the partner must also provide a short description to display
in the customer's account history.

During the processing of the transaction, the partner may provide updates
to the customer, that will appear in the customer's account history on
MtGox.

\section{API}

\subsection{User side}

\end{document}

